\documentclass[phd]{pittetd}
%=============================================%
% OPTIONS FOR THE "\DOCUMENTCLASS" DEFINITION %
%=============================================%

% Include the desired options in the documentclass (e.g., '[ma,10pt]')

% TITLE
% 1) phd = Doctor of Philosophy
% 2) ma = Master of Arts
% 3) ms = Master of Sciences
% 4) bphil = Bachelor's of Philosophy
% 5) phdprospectus = Prospectus (for Doctor of Philosoph)

% FONT SIZE
% 1) 12pt (Default, no specification needed)
% 2) 11pt
% 3) 10pt

% CHAPTER NUMBERING
% 1) default = No specification needed
%      Chapters with numbers (1.0, 2.0, etc.)
%      Sections with numbers and sub-numbers (1.1, 1.2, 2.1, 2.2, etc.)
%      Subsections with numbers and sub-numbers (an additional sub-number)
%        (1.1.1, 1.1.2, 2.1.1, 2.1.2, etc.)
%      Subsubsections with numbers and sub-numbers (two additional sub-numbers)
%        (1.1.1.1, 1.1.1.2, 2.1.1.1, 2.1.1.2, etc.)
% 2) sectionletters = Changes numbering format
%      Chapters with Roman numerals (I, II, etc.)
%      Sections with letters (A, B, etc.)
%      Subsections with numbers (1, 2, etc.)
%      Subsubsections with lowercase letters (a, b, etc.)

% SPECIAL OPTIONS
% 1) 'final' = Changes all "format warnings" into errors (Final version of the document)


%===========================%
% PACKAGES FOR THE DOCUMENT %
%===========================%

\usepackage{graphicx}       % allows images in the document
\usepackage{indentfirst}    % always indent first line in a paragraph
\usepackage{amsmath,amsthm} % better, more math typesetting
\usepackage{pdflscape}      % landscape layout page support for PDFs


%========================%
% TITLE PAGE INFORMATION %
%========================%

% Fill these in with your title page and related details

% TITLE OF THE DISSERTATION
\title[Title displayed in Acrobat Reader's Document Info Dialog Box]{Title of the Dissertation: Sample File for a Thesis with the `pittetd' Class}
% The optional (first) argument will be the title you PDF viewer uses,
% e.g.: \title[Name - Short Title]{Title of the Document}

% AUTHOR INFORMATION
\author{William Pitt}
% Previous Degree(s)
\degree{Previous degree, institution, year}
%\degree{
%  Highest previous degree, institution, year \\
%  Next highest degree, institution, year ...
%}

% DEPARTMENT INFORMATION
\school{Department of Mathematics}
% The name of the school will be prefixed with 'The' unless otherwise specified,
% e.g.: \school[certain]{department}

% DATES
% Dissertation Date (Second Page, 'ii', of the title)
% 1) Default ('today' = date of latest compilation) = No specification needed
% 2) Custom date = \date{July 20, 2017}
%\date{Month Day, Year}

% Document Year (First page, i, of the title)
% 1) Default ('current year' = year of latest compilation) = No specification needed
% 2) Custom year = \setyear{2017}
\setyear{2023}

% OPTIONS FOR THE ADOBE READER'S DOCUMENT INFO DIALOG BOX
%Include keywords
\keywords{hail-to-pitt, pittetd, theses, format}
% This list appears in your PDF metadata as 'Keywords'

% Subject
\subject{Dissertation}
% This is the 'Subject' field in your PDF metadata.


% Change Figure and Table Numeration
%\chapterfloats
% Uncomment this to get figures and tables numbered within chapters.

%======================================%
% CREATING THE DOCUMENT AND TITLE PAGE %
%======================================%

\begin{document}
\maketitle % Title must come first, immediately after \begin{document}


%=============================%
% CREATING THE COMMITTEE PAGE %
%=============================%

% For the committee membership page, you have to provide the names and affiliations of the members.

% COMMITTEE MEMBERS
% To add more committee members, simply add another \committeemember
\committeemember{Second member's name, Departmental Affiliation}
\committeemember{Third member's name, Departmental Affiliation}
% THESIS ADVISOR (First member of the committee)
\committeemember{Thesis Advisor: Advisor's name, Advisor Departmental Affiliation}
% THESIS CO-ADVISOR
%\coadvisor{Second advisor, Co-advisor Departmental Affiliation}
%Uncomment to add a 'Second Advisor' into the document

% To add more committee members
%\committeemember{Fourth member's name, Departmental Affiliation}
%\committeemember{Fifth member's name, Departmental Affiliation}
%\committeemember{Sixth member's name, Departmental Affiliation}
% To use uncommon the different committee members or add '\committeemember' commands as needed

%Special Option
% For master's theses, the committee may be omitted, naming only the advisor.

% DEPARTMENT INFORMATION
\school{Department of Mathematics}
\makecommittee


%=============================%
% CREATING THE COPYRIGHT PAGE %
%=============================%

% Create a copyright page with the author and year specified in the 'Title Page'
\copyrightpage % Uncomment to get a copyright page, comment-out to omit it.


%=======================%
% CREATING THE ABSTRACT %
%=======================%

% SPECIAL OPTIONS
% Include Keywords
% To include keywords as part of the abstract include the option '[keywords]'
% (e.g., \begin{abstract}[Keywords:])
% The list comes from the '\keywords' specified in the 'Title Page'

% To include the word 'Abstract' on the page, use '\begin{abstract*}' and '\end{abstract*}'
% instead of '\begin{abstract}' and '\end{abstract}'

\begin{abstract}
The abstract of the document.
This document is a sample file for the creation of ETD's at Pitt through \LaTeX.
\end{abstract}


%========================================%
% TABLE OF CONTENTS, FIGURES, AND TABLES %
%========================================%

% LaTeX automatically includes all the figures and tables from the figures/tables included in the
% document. If no figures and/or tables are included in the document, this template will still
% create an empty page for the figures and/or tables.

% Table of Contents
\tableofcontents

% List of Tables
\listoftables

% List of Figures
\listoffigures


%=====================%
% INCLUDING A PREFACE %
%=====================%

% To include a preface for the document uncomment the '\preface' command
% Include the text of the preface after the command

\phantomsection
\preface

% Text of the preface
This is the text of the preface, with acknowledgments, dedication, etc.


%===========================%
% START OF THE MAIN CONTENT %
%===========================%

% The document begins by specifying the first chapter with the command '\chapter{}'
\chapter{Introduction}%
We begin by saying that we do not really have much to say, but for the sake of clarity we divide our topic in chapters.

% A table example
An example of a table
\begin{table}[h]
\centering
\caption{Caption: My Table}
\label{Reference: Title of my Table}
\begin{tabular}{|c|c|c|c|}
\hline
Column 1 & Column 2 & Column 3 &\\ \hline
Row 1 & Row 1 & Row 1 & Another Row\\ \hline
Row 2 & Row 2 & Row 2 & Yup\\ \hline
\end{tabular}
\end{table}


\section{First Section}
% Remember to capitalize the sections (otherwise, the bookmark will be lowercase)
The topic treated here, given its complexity, merits an additional subdivision.


\subsection{First Subsection}
This is well-known topic, and we shall discuss it no more.
\section{Testing the template}
I am trying to show how LaTeX works.
But I don't want to start a new paragraph \\ yet.
\cite{DUMMY:1}


\chapter{Second chapter}
The topics treated in this chapter can be somewhat obscure.
For humanitarian considerations, the chapter will be subdivided.
\section{First Section}
% Cite Example
% To cite a given article contained in the 'etdbib.bib' file you can use the '\cite' command with the name of the entry in the file
This is a citation example for the Bibliography section.\cite{DUMMY:2}


\subsection{First subsection of the section}

\begin{equation} \label{EQ1}
     1 + e^{i \pi} = 0.
\end{equation}
\subsubsection{First subsubsection of the subsection}

\subsection{Second subsection of the section}
This is a very complicated topic and we shall discuss it in our next paper.\cite{DUMMY:11}
\footnote{Test}

\subsubsection{Second subsubsection of the subsection}
Lorem ipsum dolor sit amet, consectetur adipiscing elit, sed do eiusmod tempor incididunt ut labore
et dolore magna aliqua.\cite{DUMMY:3}
Ut enim ad minim veniam, quis nostrud exercitation ullamco laboris nisi ut aliquip ex ea commodo
consequat.
Duis aute irure dolor in reprehenderit in voluptate velit esse cillum dolore eu fugiat nulla
pariatur.\cite{DUMMY:4}
Excepteur sint occaecat cupidatat non proident, sunt in culpa qui officia deserunt mollit anim id
est laborum.\cite{DUMMY:5}


\chapter{Conclusions}
This is the third chapter of the present dissertation.\cite{DUMMY:6}
It is more interesting than the first two, for it is the last one.\cite{DUMMY:7}

Lorem ipsum dolor sit amet, consectetur adipiscing elit, sed do eiusmod tempor incididunt ut labore
et dolore magna aliqua.\cite{DUMMY:8}
Ut enim ad minim veniam, quis nostrud exercitation ullamco laboris nisi ut aliquip ex ea commodo
consequat.\cite{DUMMY:9}
Duis aute irure dolor in reprehenderit in voluptate velit esse cillum dolore eu fugiat nulla
pariatur.\cite{DUMMY:10}
Excepteur sint occaecat cupidatat non proident, sunt in culpa qui officia deserunt mollit anim id
est laborum.


%==========%
% APPENDIX %
%==========%
\appendix % After this command, chapters will be formatted as appendices.

\chapter{Examples and Results}
\begin{figure}[t]
    \centering
    \includegraphics[width=0.9\textwidth]{Images/Picture-Example.jpg}
    \caption{Caption: Image Example}
    \label{Reference: Picture Example}
\end{figure}

\chapter{Second Appendix}


%==============%
% BIBLIOGRAPHY %
%==============%
\safebibliography{etdbib.bib}
\bibliographystyle{plain}

\end{document}
